\section{Introduction}
% Text from proposal follows
In the Department of Nuclear, Plasma, and Radiological Engineering at the 
University of Illinois, we have published multiple energy analysis studies of 
this kind \cite{stuff}. Most recently, we conducted a detailed techno-economic 
optimization of the UIUC campus microgrid as part of a DOE-NE award. We propose 
to expand this UIUC microgrid optimization to capture the state of Illinois by 
incorporating data directly from the Energy Information Administration, the 
Illinois Department of Employment Security, and previous reports on this topic 
(e.g. \cite{stuff}).

Our current model leverages a robust dataset that populates multiple analytic 
models, trained neural networks, and  the Temoa framework (Tools for Energy 
Model Optimization and Analysis). Temoa is an open-source framework for 
simulating and optimizing energy systems. With our UIUC microgrid model in 
Temoa (Figure 1), we are able to determine the optimal energy generation mix 
for the UIUC microgrid for various objective functions (minimize carbon, 
minimize cost, etc.) while meeting system constraints (zero carbon by 2050, 
practical deployment speeds, realistic improvements in storage technology 
capabilities, etc.). The UIUC microgrid model includes transportation as well 
as campus steam, electric, and chilled water demand and provides cross-decadal 
deployment solutions which optimize the scenario objective function (Figure 2).

