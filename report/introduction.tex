\section{Introduction}
% Text from proposal follows
Eleven (11) carbon free nuclear power reactors at 6 sites produce the majority of
electricity in Illinois and critically underpin its clean energy future. This
report quantitatively demonstrates the role nuclear energy must play in
minimizing cost while meeting Illinois’ carbon goals
through 2050. We have conducted techno-economic optimizations of
the Illinois energy system to establish optimal energy generation scenarios with and without the
at-risk plants. With these, we compared and contrasted the economic and carbon implications of
these energy futures. In addition to emissions, this report also considers
other environmental impacts of available energy choices, such as land use and
waste burdens.

Two reactors at the Byron Generating Station together generate over 2 GWe of
emissions-free nuclear energy. This plant provides the Byron, Illinois region
with hundreds of skilled jobs and \$38 million annually in taxes.

% from sam's paper
Baker et. al found that increasing penetration of renewables increases grid 
volatility which exponentially increased the \gls{LCOE} of the energy system when 
subject to a mismatch penalty. They also found that energy storage becomes 
increas- ingly valuable for grid stability at or above 20\% renewable 
penetration.

