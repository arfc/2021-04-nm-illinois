\section{Introduction}\label{sec:intro}
Eleven (11) emissions-free nuclear reactors at six (6) sites produce the majority of
electricity in Illinois and critically underpin its clean energy future. Four 
(4) of these reactors, representing over 4GWe of electric capacity, are at the Byron 
and Dresden plants, which face premature closure. This
report quantitatively demonstrates the role nuclear energy must play in
minimizing cost while meeting Illinois’ carbon goals through 2050, with a 
particular focus on those plants. 

We have 
modeled the Illinois electric grid and conducted optimization simulations of 
key policy scenarios. These simulations establish the least costly energy 
generation mixtures with and without the at-risk plants in the context of 
various policy factors, such as zero-emissions targets. 
With these solutions, we compared the economic and carbon implications of these energy futures. 
In addition to emissions, this report also considers
other environmental impacts of available energy choices, such as land use and
solid waste generation. Other recent work has reviewed the potential health 
impacts of these closures \cite{catf_potential_2021}.

We built a computational model of Illinois' electric system that leverages high 
fidelity data from a variety of sources to explore various potential policy 
scenarios in the 2020-2050 time frame.  Comparison among optimal solutions 
quantified the economic and emissions impacts of prematurely closing nuclear 
plants, capping emissions, aggressively installing renewable generation, and 
deploying advanced nuclear reactors.  The following sections describe the 
methods, data, and assumptions used in the modeled scenarios (Section 
\ref{sec:methods}), the resulting optimal solutions (Section 
\ref{sec:results}), and a discussion of the key findings (Section 
\ref{sec:discussion}).  
