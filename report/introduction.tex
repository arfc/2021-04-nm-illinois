\section{Introduction}
% Text from proposal follows
In the Department of Nuclear, Plasma, and Radiological Engineering at the 
University of Illinois, we have published multiple energy analysis studies of 
this kind \cite{stuff}. Most recently, we conducted a detailed techno-economic 
optimization of the UIUC campus microgrid as part of a DOE-NE award. We propose 
to expand this UIUC microgrid optimization to capture the state of Illinois by 
incorporating data directly from the Energy Information Administration, the 
Illinois Department of Employment Security, and previous reports on this topic 
(e.g. \cite{stuff}).

Our current model leverages a robust dataset that populates multiple analytic 
models, trained neural networks, and  the Temoa framework (Tools for Energy 
Model Optimization and Analysis)\cite{temoa}. Temoa is an open-source framework for 
simulating and optimizing energy systems developed at North Carolina State 
University. With our \gls{UIUC} microgrid model in 
Temoa (Figure 1), we are able to determine the optimal energy generation mix 
for the \gls{UIUC} microgrid for various objective functions (minimize carbon, 
minimize cost, etc.) while meeting system constraints (zero carbon by 2050, 
practical deployment speeds, realistic improvements in storage technology 
capabilities, etc.). The UIUC microgrid model includes transportation as well 
as campus steam, electric, and chilled water demand and provides cross-decadal 
deployment solutions which optimize the scenario objective function.



Eleven (11) carbon free nuclear power reactors at 6 sites produce the majority of
electricity in Illinois and critically underpin its clean energy future. This
report quantitatively demonstrates the role nuclear energy must play in
minimizing cost while meeting Illinois’ carbon goals
through 2050. We have conducted techno-economic optimizations of
the Illinois energy system to establish optimal energy generation scenarios with and without the
at-risk plants. With these, we compared and contrasted the economic and carbon implications of
these energy futures. In addition to emissions, this report also considers
other environmental impacts of available energy choices, such as land use and
waste burdens.

Two reactors at the Byron Generating Station together generate over 2 GWe of
emissions-free nuclear energy. This plant provides the Byron, Illinois region
with hundreds of skilled jobs and \$38 million annually in taxes.
