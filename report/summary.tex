\section{Summary}
On August 27, 2020, Exelon Generation announced planned premature closures of 
two Illinois nuclear plants (4 reactor units), which compete economically with 
fossil fueled plants within the \gls{PJM} interconnection \cite{larson_exelon_2020}. 
This report quantitatively explores how these closures would undermine economic 
and decarbonization goals in the state of Illinois, such as an aggressive 
target to achieve a zero carbon electric grid by 2030.

Previous energy systems research has shown that such clean energy goals cannot 
be reached if nuclear plants prematurely retire 
\cite{national_academies_of_sciences_engineering_and_medicine_2021_accelerating_2021,larson_net-zero_2020,davis_net-zero_2018}.  
In particular, the February 2021 National Academy of Sciences, Engineering, and 
Medicine consensus report, ``Accelerating Decarbonization of the U.S. Energy 
System,'' determined unequivocally that U.S. decarbonization will require keeping 
existing nuclear plants open 
\cite{national_academies_of_sciences_engineering_and_medicine_2021_accelerating_2021}. 
Consistent with that literature, our simulations indicate that decarbonization 
in Illinois will require not only maintenance but expansion of nuclear energy capacity. 
\textbf{The simulations in this report minimize future Illinois electric system 
cost in the context of potential policy constraints and demonstrate that:}

\begin{itemize}
        \item nuclear energy is necessary to reach Illinois' carbon reduction 
                goals;
        \item without existing nuclear power, reaching zero carbon would 
                require solar deployments to displace $10,000km^2$ of critical 
                Illinois 
                farmland;
        \item and deploying new advanced nuclear generation is the least expensive way to 
                allow Illinois farmland to remain farmland while reaching 
                zero-carbon by 2030.
  \end{itemize}

\textbf{These simulations also revealed many specific, complementary 
conclusions, such as:}
\begin{itemize}
        \item Keeping Illinois' existing nuclear plants open through 
                2050 avoids 25 million metric tons of life-cycle CO$_2$ 
                emissions and 600,000 metric tons of e-waste.
        \item Even if advanced nuclear deployments experienced 200\% capital 
                cost overruns, total system cost impacts would be negligible.
        \item Deploying advanced nuclear avoids approximately 900,000 metric 
                tons of e-waste. 
        \item Significant, possibly infeasible, grid-scale battery storage capacity 
                is required to meet any zero-carbon target with renewable 
                penetration.
\end{itemize}

