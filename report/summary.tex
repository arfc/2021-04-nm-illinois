\section{Summary}

% These are draft comments, various summary sentences to frame the gist.
Exelon Generation annouced planned premautre closures of two nuclear plants, .  
Byron and Dresden. 
Both plants, within the PJM interconnection \todo{double check} have suffered 
under the capacity market situation.
Nuclear is carbon free, high energy density, etc.
This report explores the extent to which economic and decarbonization goals in 
the state of Illinois will be undermined by these closures.  
Clean energy goals will not be reached if these plants prematurely retire. 
Additionally, the economic value provided by these plants to the surrounding community will be lost. 

Eleven (11) carbon free nuclear power reactors at 6 sites produce the majority of 
electricity in Illinois and critically underpin its clean energy future. This 
report quantitatively demonstrates the role nuclear energy must play in 
maximizing job creation, minimizing cost, and meeting Illinois’ carbon goals 
through 2050. We have conducted a 50-year techno-economic optimization of 
the Illinois energy system to analyze scenarios with and without the 
at-risk plants. With these, we compared and contrasted the economic and carbon implications of 
these energy futures. 


Statements from the ANS students
%If Dresden and Byron close:
%Illinois will see a 30\% hit to carbon-free electricity
Our regional findings are consistent with the February 2021 National Academy of 
Sciences, Engineering, and Medicine report, “Accelerating Decarbonization of 
the U.S. Energy System”, which determined unequivocally that US decarbonization 
will require keeping existing nuclear plants open 
\cite{national_accelerating_2021}.


Describe the simulations at a high level.
Describe the major assumptions.
(how did we model the carbon emissions, what generation did we assume would 
replace these plants, etc.)

%nice text from the ANS students
%Exelon Generation plans to close Byron and Dresden Nuclear Plants in the fall of 
%2021 closure of these plants is driven by concern over their
%economic competitiveness in the current electricity market. However, nuclear power plants
%provide benefits that are not often captured in the revenue they generate, such as a steady
%supply of carbon-free electricity, a large tax base to the state and local community, and
%hundreds of permanent, well-paying jobs.
%wind and solar will not be able to make up this  gap
%we will have to start using natural gas \& coal in their place
%Thousands of people’s jobs will be at risk 
%Including current University of Illinois students
%Communities will lose a huge source of funding for them 
%Nuclear power plants in Illinois pay a lot of property taxes that fund schools and communities
%Exelon has a culture of volunteering and supporting the community
%Intern Service Projects
%Quad Cities Interns Raised \$34,100 for River Bend Food Bank and the Quad Cities Disaster Recovery Fund
%4 other at-risk plants
%Here are some contributions made by those 4 plants  
%\$3.5 billion to state GDP
%28,000 in-state jobs (direct & secondary)
%Maintain \$149 million in annual state tax revenue 
%Avoid 45 million metric tons of CO2 emissions each year
%Illinois consumers and businesses would pay about \$483 million 
%more annually for electricity without these four plants

%In Illinois, we enjoy a remarkably clean energy mix thanks to our nuclear power 
%plants. The six modestly-sized plants produce 90\% of the clean electricity in 
%Illinois and over half of our total electricity  generation. They run reliably 
%day and night, unaffected by weather. They also require a remarkably small 
%footprint, and their waste is tiny in volume, carefully managed and presents no 
%risks to the public, now or in the future. These nuclear plants are the crucial 
%ingredient in any low-carbon industrial future for the state, and are 
%threatened with shutdown decades before any need to consider decommissioning 
%for technical reasons. 

%Luckily, Illinois’ nuclear plants are among the best in the world, and there is 
%no reason they shouldn’t continue to run for decades to come. Nuclear plants of 
%similar design and vintage around the world are seeing the lives of their 
%reactors extended to 80 years, with no limitation in sight. Further, with 
%proper maintenance and part replacement, these plants can continue to operate 
%like new even further beyond that. 
%If these nuclear plants are forced to close, and they will without action from 
%the state or the electricity market, they will be replaced by natural gas, not 
%solar and wind. This has been the result in Vermont, California, Massachusetts 
%and just last year in New York with the closure of Indian Point. Not only would 
%this be bad for emissions goals, but also for consumers who will pay \$480 
%million per year in higher rates, according to a recent report from Brattle. If 
%these nuclear plants close, citizens of Illinois will be forced to pay more 
%money to support the operation of fossil fuel plants in Indiana and Ohio that 
%don't protect the environment or pay for their own carbon emissions. 
%The premature closure of the state’s nuclear plants will especially impact the 
%towns and communities that rely on them. Nuclear plants are economic engines 
%that provide permanent, high-wage, high-skilled jobs for generations. And for 
%every 100 nuclear power plant jobs, 66 more jobs are created in local 
%communities. Replacing in-state nuclear generation with out-of-state natural 
%gas will result in a net job loss for Illinois.  


