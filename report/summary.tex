\section{Summary}
% These are draft comments, various summary sentences to frame the gist.
On August 27, 2020, Exelon Generation annouced planned premautre closures of two nuclear plants, 
Byron and Dresden \cite{larson_exelon_2020}. These premature closures reflect 
the economic challenges these plants face in competition with fossil fueled 
plants in the market forces within the PJM interconnection. Specifically, in PJM capacity 
auctions,  low-cost fossil fueled plants, particularly natural gas generation, 
have recently successfully underbid emissions-free nuclear energy. 
Corresponding revenue shortfalls in the hundreds of millions of dollars
undermine the economic feasibility of these plants. 

This report explores the extent to which these closures would undermine 
economic and decarbonization goals in the state of Illinois.  Specifically, 
proposals in Illinois include agressive carbon reduction targets (namely, zero carbon by 2030) 
that would necessitate not only maintenance but expansion of firm, 
emissions-free baseload energy capacity. Energy analysis literature 
(\cite{national_accelerating_2021,others}) implies that those clean energy 
goals will not be reached if these nuclear plants prematurely retire and that 
the economic value provided by these plants to the surrounding community will 
be lost \cite{economic_impact_studies}. 
Our findings are consistent with that literature, particularly the February 
2021 National Academy of Sciences, Engineering, and Medicine report, 
``Accelerating Decarbonization of the U.S. Energy System,'' which determined 
unequivocally that US decarbonization will require keeping existing nuclear 
plants open \cite{national_accelerating_2021}.

\textbf{The simulations in this report minimize future Illinois electric system cost 
in the context of constraints imposed by various policy futures.} These simulations utilized the 
Temoa tool from \gls{NCSU} for Techno-economic modeling, optimization, and 
analysis. We primarily leveraged data from \gls{EIA}, \gls{IEA}, 
\gls{NREL}, and \gls{NEI}. Based on this framework, this work explored the economic 
and emissions impacts of prematurely closing nuclear plants, capping 
emissions, pricing carbon, agressivly installing renewable generation, and deploying advanced nuclear plants.

The energy mixtures that minimized cost in each scenario demonstrated that nuclear energy is necessary to 
reach Illinois' carbon reduction goals unless nearly all farmland in 
Illinois were dedicated to solar and wind deployments. 
Keeping all of Illinois' existing nuclear plants open through 2050 avoids 25 
million metric tons of lifecycle CO$^2$ emissions. 
Additionally, a carbon-free grid in Illinois will require significant 
grid-scale battery storage,  which meets the strategic reserve requirement of 35\%.
Keeping all Illinois' existing nuclear plants open through 2050 would reduce 
e-waste by 600,000 metric tons and deploying advanced nuclear avoids 
approximately 900,000 metric tons of e-waste. 

Without preserving existing nuclear and deploying advanced reactors, the 
required land use for solar and wind generation is 
unattainable, particularly since the Illinois land appropriate for wind and solar 
is already in use as the vital farmland the nation currently relies on for \todo{percent} of its 
corn and \todo{percent} of its soybeans.
Specifically, strategies which allow nuclear plants to close before 2050 
require approximately 12\% of the land area of Illinois to be devoted to 
renewable generation \todo{this would be better broken down into types of 
generation\ldots } with  4\% of Illinois' land area in use for rooftop solar. 
Keeping the nuclear plants open through 2050 halves this requirement.

Finally, simulations exploring the possible impact of cost overruns in advanced nuclear power 
plant deployment showed that even 200\% capital cost overruns in advanced 
nuclear deployments would have a negligible impact on total system cost.
That is, even if advanced nuclear were twice as expensive as predicted, it 
would still be wise to deploy it in Illinois by 2030, given the state's carbon 
reduction goals.

%nice text from the ANS students
%Exelon Generation plans to close Byron and Dresden Nuclear Plants in the fall of 
%2021 closure of these plants is driven by concern over their
%economic competitiveness in the current electricity market. However, nuclear power plants
%provide benefits that are not often captured in the revenue they generate, such as a steady
%supply of carbon-free electricity, a large tax base to the state and local community, and
%hundreds of permanent, well-paying jobs.
%wind and solar will not be able to make up this  gap
%we will have to start using natural gas \& coal in their place
%Thousands of people’s jobs will be at risk 
%Including current University of Illinois students
%Communities will lose a huge source of funding for them 
%Nuclear power plants in Illinois pay a lot of property taxes that fund schools and communities
%Exelon has a culture of volunteering and supporting the community
%Intern Service Projects
%Quad Cities Interns Raised \$34,100 for River Bend Food Bank and the Quad Cities Disaster Recovery Fund
%4 other at-risk plants
%Here are some contributions made by those 4 plants  
%\$3.5 billion to state GDP
%28,000 in-state jobs (direct & secondary)
%Maintain \$149 million in annual state tax revenue 
%Avoid 45 million metric tons of CO2 emissions each year
%Illinois consumers and businesses would pay about \$483 million 
%more annually for electricity without these four plants

%In Illinois, we enjoy a remarkably clean energy mix thanks to our nuclear power 
%plants. The six modestly-sized plants produce 90\% of the clean electricity in 
%Illinois and over half of our total electricity  generation. They run reliably 
%day and night, unaffected by weather. They also require a remarkably small 
%footprint, and their waste is tiny in volume, carefully managed and presents no 
%risks to the public, now or in the future. These nuclear plants are the crucial 
%ingredient in any low-carbon industrial future for the state, and are 
%threatened with shutdown decades before any need to consider decommissioning 
%for technical reasons. 

%Luckily, Illinois’ nuclear plants are among the best in the world, and there is 
%no reason they shouldn’t continue to run for decades to come. Nuclear plants of 
%similar design and vintage around the world are seeing the lives of their 
%reactors extended to 80 years, with no limitation in sight. Further, with 
%proper maintenance and part replacement, these plants can continue to operate 
%like new even further beyond that. 
%If these nuclear plants are forced to close, and they will without action from 
%the state or the electricity market, they will be replaced by natural gas, not 
%solar and wind. This has been the result in Vermont, California, Massachusetts 
%and just last year in New York with the closure of Indian Point. Not only would 
%this be bad for emissions goals, but also for consumers who will pay \$480 
%million per year in higher rates, according to a recent report from Brattle. If 
%these nuclear plants close, citizens of Illinois will be forced to pay more 
%money to support the operation of fossil fuel plants in Indiana and Ohio that 
%don't protect the environment or pay for their own carbon emissions. 
%The premature closure of the state’s nuclear plants will especially impact the 
%towns and communities that rely on them. Nuclear plants are economic engines 
%that provide permanent, high-wage, high-skilled jobs for generations. And for 
%every 100 nuclear power plant jobs, 66 more jobs are created in local 
%communities. Replacing in-state nuclear generation with out-of-state natural 
%gas will result in a net job loss for Illinois.  


