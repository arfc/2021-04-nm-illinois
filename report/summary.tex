\section{Summary}
% These are draft comments, various summary sentences to frame the gist.
On August 27, 2020, Exelon Generation annouced planned premautre closures of two nuclear plants, 
Byron and Dresden \cite{larson_exelon_2020}. These premature closures reflect 
the economic challenges these plants face in competition with fossil fueled 
plants in the market forces within the PJM interconnection. Specifically, in PJM capacity 
auctions,  low-cost fossil fueled plants, particularly natural gas generation, 
have recently successfully underbid emissions-free nuclear energy. 
Corresponding revenue shortfalls in the hundreds of millions of dollars
undermine the economic feasibility of these plants. 


This report explores the extent to which economic and decarbonization goals in
the state of Illinois will be undermined by these closures. Specifically, the
Illinois state legislature is considering agressive carbon reduction goals that
would necessitate not only maintenance but expansion of firm, emissions-free
baseload energy capacity.  Clean energy goals will not be reached if these
plants prematurely retire.  Additionally, the economic value provided by these
plants to the surrounding community will be lost. 

Our regional findings are consistent with the February 2021 National Academy of 
Sciences, Engineering, and Medicine report, ``Accelerating Decarbonization of 
the U.S. Energy System,'' which determined unequivocally that US decarbonization 
will require keeping existing nuclear plants open 
\cite{national_accelerating_2021}.


Describe the simulations at a high level.
Describe the major assumptions.
(how did we model the carbon emissions, what generation did we assume would 
replace these plants, etc.)

These simulations explored business 
Business as usual (premature closure, no carbon cap)
SD2 (scheduled closure, no carbon cap)
SD3 (nuclear closes after 2050, no carbon cap)
SD4 (premature closure, with carbon cap)
SD5 (scheduled closure, with carbon cap)
SD6 (nuclear closes after 2050, with carbon cap)
--- Simulating aggressive cost overruns (200%) for advanced nuclear projects (eXpensive Nuclear)
XN1 (premature closure, with carbon cap, advanced nuclear is twice as expensive)
XN2 (scheduled closure, with carbon cap, advanced nuclear is twice as expensive)
XN3 (existing nuclear closes after 2050, with carbon cap, advanced nuclear is twice as expensive)
--- Simulating zero advanced nuclear allowed (Zero Nuclear)
ZN1 (premature closure, with carbon cap)
ZN2 (scheduled closure, with carbon cap)
ZN1 (existing nuclear closes after 2050, with carbon cap)
Every scenario has a required battery storage applied Section \ref{sec:storage} 
and battery waste accumulated in Section \ref{sec:waste}.

Basic constraints;
In SDx and XNx

Utility scale solar is restricted to 10 GW by 2030, reflecting an aggressive and optimistic build out based on CEJA goals.
Wind turbines are restricted to 13.8 GW by 2030, reflecting an aggressive and optimistic build out based on CEJA goals.
Residential solar is allowed to increase at a steady rate, but is capped at 75% of the technical resource availability, determined from a GIS analysis by NREL (i.e. 75% of the theoretical limit).
In ZNx

The constraints on utility scale wind and solar are lifted. It is not possible to achieve zero carbon without advanced nuclear under the above constraints.
Key observations:

Increasing the investment cost for advanced nuclear by 200% has an almost negligible impact on total system cost.
Zero advanced nuclear scenarios are technically less costly, but require an unimaginably aggressive build of solar by 2030 (56 GW!!)
Zero advanced nuclear scenarios require about 12% land use change for renewables (4% land area for rooftop solar). Keeping the nuclear plants open through 2050 halves this requirement (since nuclear is half of IL generation, currently).
Keeping the nuclear plants open through 2050 would reduce e-waste by 600,000 metric tons. Using advanced nuclear saves about 900,000 metric tons.
Keeping the nuclear plants open through 2050 avoids 25 million metric tons of lifecycle CO2 emissions if advanced nuclear is used, and 30 million metric tons if majority renewables is used. Pursuing advanced nuclear over renewables saves about 5 million metric tons.

%nice text from the ANS students
%Exelon Generation plans to close Byron and Dresden Nuclear Plants in the fall of 
%2021 closure of these plants is driven by concern over their
%economic competitiveness in the current electricity market. However, nuclear power plants
%provide benefits that are not often captured in the revenue they generate, such as a steady
%supply of carbon-free electricity, a large tax base to the state and local community, and
%hundreds of permanent, well-paying jobs.
%wind and solar will not be able to make up this  gap
%we will have to start using natural gas \& coal in their place
%Thousands of people’s jobs will be at risk 
%Including current University of Illinois students
%Communities will lose a huge source of funding for them 
%Nuclear power plants in Illinois pay a lot of property taxes that fund schools and communities
%Exelon has a culture of volunteering and supporting the community
%Intern Service Projects
%Quad Cities Interns Raised \$34,100 for River Bend Food Bank and the Quad Cities Disaster Recovery Fund
%4 other at-risk plants
%Here are some contributions made by those 4 plants  
%\$3.5 billion to state GDP
%28,000 in-state jobs (direct & secondary)
%Maintain \$149 million in annual state tax revenue 
%Avoid 45 million metric tons of CO2 emissions each year
%Illinois consumers and businesses would pay about \$483 million 
%more annually for electricity without these four plants

%In Illinois, we enjoy a remarkably clean energy mix thanks to our nuclear power 
%plants. The six modestly-sized plants produce 90\% of the clean electricity in 
%Illinois and over half of our total electricity  generation. They run reliably 
%day and night, unaffected by weather. They also require a remarkably small 
%footprint, and their waste is tiny in volume, carefully managed and presents no 
%risks to the public, now or in the future. These nuclear plants are the crucial 
%ingredient in any low-carbon industrial future for the state, and are 
%threatened with shutdown decades before any need to consider decommissioning 
%for technical reasons. 

%Luckily, Illinois’ nuclear plants are among the best in the world, and there is 
%no reason they shouldn’t continue to run for decades to come. Nuclear plants of 
%similar design and vintage around the world are seeing the lives of their 
%reactors extended to 80 years, with no limitation in sight. Further, with 
%proper maintenance and part replacement, these plants can continue to operate 
%like new even further beyond that. 
%If these nuclear plants are forced to close, and they will without action from 
%the state or the electricity market, they will be replaced by natural gas, not 
%solar and wind. This has been the result in Vermont, California, Massachusetts 
%and just last year in New York with the closure of Indian Point. Not only would 
%this be bad for emissions goals, but also for consumers who will pay \$480 
%million per year in higher rates, according to a recent report from Brattle. If 
%these nuclear plants close, citizens of Illinois will be forced to pay more 
%money to support the operation of fossil fuel plants in Indiana and Ohio that 
%don't protect the environment or pay for their own carbon emissions. 
%The premature closure of the state’s nuclear plants will especially impact the 
%towns and communities that rely on them. Nuclear plants are economic engines 
%that provide permanent, high-wage, high-skilled jobs for generations. And for 
%every 100 nuclear power plant jobs, 66 more jobs are created in local 
%communities. Replacing in-state nuclear generation with out-of-state natural 
%gas will result in a net job loss for Illinois.  


