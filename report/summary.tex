\section{Summary}
On August 27, 2020, Exelon Generation announced planned premature closures of 
two Illinois nuclear plants (4 reactor units) which compete economically with 
fossil fueled plants within the \gls{PJM} interconnection \cite{larson_exelon_2020}. 
This report quantitatively explores how these closures would undermine economic 
and decarbonization goals in the state of Illinois, such as an aggressive 
target to achieve a zero carbon electric grid by 2030.

Previous energy systems research has shown that such clean energy goals cannot 
be reached if nuclear plants prematurely retire 
\cite{national_academies_of_sciences_engineering_and_medicine_2021_accelerating_2021,larson_net-zero_2020,davis_net-zero_2018}.  
In particular, the February 2021 National Academy of Sciences, Engineering, and 
Medicine consensus report, ``Accelerating Decarbonization of the U.S. Energy 
System,'' determined unequivocally that US decarbonization will require keeping 
existing nuclear plants open 
\cite{national_academies_of_sciences_engineering_and_medicine_2021_accelerating_2021}. 
Consistent with that literature, our simulations indicate that decarbonization 
in Illinois will require not only maintenance but expansion of nuclear energy capacity. 

\textbf{The simulations in this report minimize future Illinois electric system 
cost in the context of potential policy constraints.} 
These optimizations demonstrated that:


\begin{itemize}
        \item nuclear energy is necessary to reach Illinois' carbon reduction 
                goals,
        \item without existing nuclear power, reaching zero carbon would 
                require solar deployments to displace $10,000km^2$ of critical 
                Illinois 
                farmland.
        \item and deploying new advanced nuclear generation is the least expensive way to 
                allow Illinois farmland to remain farmland while reaching 
                zero-carbon by 2030.
  \end{itemize}

The simulations revealed many specific conclusions in support of these lessons, 
such as:
\begin{itemize}
        \item Keeping Illinois' existing nuclear plants open through 
                2050 avoids 25 million metric tons of life-cycle CO$_2$ 
                emissions and 600,000 metric tons of e-waste.
        \item Even if advanced nuclear deployments experienced 200\% capital 
                cost overruns, total system cost impacts would be negligible.
        \item Deploying advanced nuclear avoids approximately 900,000 metric 
                tons of e-waste. 
        \item Significant grid-scale battery storage is required to meet any zero-carbon 
                target.
\end{itemize}


%That is, even if advanced nuclear were twice as expensive as predicted, it 
%would still be wise to deploy it in Illinois by 2030, given the state's carbon 
%reduction goals.

%Luckily, Illinois’ nuclear plants are among the best in the world, and there is 
%no reason they shouldn’t continue to run for decades to come. Nuclear plants of 
%similar design and vintage around the world are seeing the lives of their 
%reactors extended to 80 years, with no limitation in sight. Further, with 
%proper maintenance and part replacement, these plants can continue to operate 
%like new even further beyond that. 
%If these nuclear plants are forced to close, and they will without action from 
%the state or the electricity market, they will be replaced by natural gas, not 
%solar and wind. This has been the result in Vermont, California, Massachusetts 
%and just last year in New York with the closure of Indian Point. Not only would 
%this be bad for emissions goals, but also for consumers who will pay \$480 
%million per year in higher rates, according to a recent report from Brattle. If 
%these nuclear plants close, citizens of Illinois will be forced to pay more 
%money to support the operation of fossil fuel plants in Indiana and Ohio that 
%don't protect the environment or pay for their own carbon emissions. 
%The premature closure of the state’s nuclear plants will especially impact the 
%towns and communities that rely on them. Nuclear plants are economic engines 
%that provide permanent, high-wage, high-skilled jobs for generations. And for 
%every 100 nuclear power plant jobs, 66 more jobs are created in local 
%communities. Replacing in-state nuclear generation with out-of-state natural 
%gas will result in a net job loss for Illinois.  


% specifically capacity auctions, have exacerbated the low-cost fossil fueled 
% plants have underbid emissions-free nuclear energy. Corresponding revenue 
% shortfalls in the hundreds 
%of millions of dollars undermined the economic feasibility of the Byron 
%and Dresden generating etations in northern Illinois, which provide over 4GWe 
%of firm, emissions-free power.
