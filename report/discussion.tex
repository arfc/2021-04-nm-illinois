\section{Discussion}
This work constructed a techno-economic model of the Illinois electric grid 
using the Temoa framework \cite{decarolis_modelling_2016}. With this framework, 
we simulated twelve (12) potential economic and policy futures for this energy 
system spanning the 2020-2050 timeframe. The linear programming model 
identified energy mixtures which minimized total system costs 
in the context of those potential technology, economic, and policy constraints.

This work accordingly adds to the growing body of research demonstrating how 
decommissioning existing, emissions-free nuclear power plants endangers the 
feasibility of near-term zero-emissions targets.  Our conclusions are 
consisistent with and confirmatory of such literature, in particular, the 
February 2021 National Academy of Sciences, Engineering, and Medicine consensus 
report, ``Accelerating Decarbonization of the U.S. Energy System,'' which determined unequivocally that US decarbonization will require keeping existing nuclear plants open
\cite{national_academies_of_sciences_engineering_and_medicine_2021_accelerating_2021}.

Specifically, our simulations indicate that decarbonization
in Illinois will require not only maintenance but expansion of nuclear energy capacity.
When the 2020-2050 cost of the Illinois electric system is minimized, 
comparison of these twelve (12) scenarios showed that :

\begin{itemize}
        \item Nuclear energy is necessary to reach Illinois' carbon reduction
                goals.
        \item Without existing nuclear power, reaching zero carbon would
                require solar deployments to displace $10,000km^2$ of critical
                Illinois
                farmland.
        \item Deploying new advanced nuclear generation is the least expensive way to
                allow Illinois farmland to remain farmland while reaching
                zero-carbon by 2030.
        \item Keeping Illinois' existing nuclear plants open through
                2050 avoids 25 million metric tons of life-cycle CO$_2$
                emissions and 600,000 metric tons of e-waste.
        \item Even if advanced nuclear deployments experienced 200\% capital
                cost overruns, total system cost impacts would be negligible.
        \item Deploying advanced nuclear avoids approximately 900,000 metric
                tons of e-waste.
        \item Significant grid-scale battery storage is required to meet any zero-carbon
                target.
\end{itemize}


